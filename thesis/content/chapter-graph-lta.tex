%!TEX root = ../main.tex
% chktex-file 46
\chapter{Learning to Aggregate on Graphs}%
\label{sec:ltag}

In the previous chapter an introduction to two separate fields of research was given:
\begin{enumerate*}
	\item \Acf{lta},
	\item \Acf{gcr}
\end{enumerate*}.
In this chapter we will combine them and define an extension of \ac{lta} to the \ac{gcr} problem.
This will be done in three steps:
\begin{enumerate}
	\item We begin with a formal definition of what actually constitutes an \ac{lta} method as opposed to non-\acs{lta} methods.
	\item Using this definition, we will see that some of the previously described \ac{gcnn} methods can be interpreted as \ac{lta} variants under certain conditions.
	\item Finally a new \acs{lta}-inspired \ac{gcnn} architecture will be described.
\end{enumerate}

\section{A Generalized Definition of \acs*{lta}}%
\label{sec:ltag:definition}

In order to formally define \ac{lta}, we must first decide on its defining characteristic.
We propose that this characteristic should be the \textit{localized explainability} of \ac{lta} predictions.
As seen in \cref{sec:related:lta}, an \ac{lta} prediction $\hat{y}_{C} \in \mathcal{Y}$ for some multiset composition $C = \ldblbrace c_1, \dots, c_n \rdblbrace$ can always be tracked back to a set of local constituent predictions $\hat{y}_1, \dots, \hat{y}_n \in \mathcal{Y}$.
Under the assumption that each constituent $c_i$ represents some human interpretable object, a composition's prediction $\hat{y}_{C}$ can therefore be explained by the presence of the constituents/objects whose local predictions $\hat{y}_i$ are indicative of the global prediction.

Based on this intuition we now give a generalized definition of \ac{lta} which applies to unstructured as well as structured input data.
We assume that compositions are represented by graphs $G \in \mathcal{G}$;
unstructured inputs are represented by graphs with one vertex per constituent ($\mathcal{V}_G = \{ v_{c_1}, \dots, v_{c_n} \}$) and no edges ($\mathcal{E}_G = \emptyset$).
Each composition has some target value $y_G \in \mathcal{Y}$ which could be a discrete class or continuous score.
An \textit{\ac{lta} model} $h: \mathcal{G} \to \mathcal{Y}$ assigns predictions $\hat{y}_G \in \mathcal{Y}$ to compositions $G$.
Such a model must satisfy three criteria:
\begin{enumerate}[label=\textbf{\arabic*.}]
	\item \textbf{Decomposition:}
		A given composition $G$ must be decomposed into a set of constituents $c_{G,i}$ via a decomposition function $\varphi: \mathcal{G} \to \mathcal{P}(\mathcal{G})$ for which it holds that $\forall G \in \mathcal{G}: \forall c_{G,i} \in \varphi(G): \exists s \in \mathcal{V}_G^{*}: c_{G,i} \equiv G[s]$.
		The decomposition function splits $G$ into its constituents $c_{G,i}$ which must be subgraphs of $G$.

		In the existing unstructured \ac{lta} approaches the decomposition function is implicitly defined as $\varphi(G) \coloneqq {\{ G[v_{c_i}] \}}_{v_{c_i} \in \mathcal{V}_G}$ since each vertex $v_{c_i}$ corresponds to an interpretable constituent $c_i$ by definition.
		For structured data however, a split into individual vertices is typically not appropriate.
		Molecular graphs from chemical datasets for example are meaningfully characterized by the presence of so-called \textit{functional groups} consisting of multiple bonded atoms while a characterization on the level of individual atoms is generally less meaningful~\cite{McNaught1997}.
	\item \textbf{Disaggregation:}
		The constituents $c_{G, i} \in \varphi(G)$ must be evaluated via some function $f: \mathcal{G} \to \mathcal{Y} \times \mathbb{R}$.
		This evaluation function assigns a prediction $\hat{y}_{G, i} \in \mathcal{Y}$ and a weight $w_{G, i} \in \mathbb{R}$ to each constituent.
		A constituent's weight $w_{G, i}$ can intuitively be interpreted as a measure of the confidence that the local prediction $\hat{y}_{G, i}$ is indicative of the composition's global target value $y_G$.
		Learning local predictions and weights for all possible constituents is called the \textit{disaggregation problem}.

		Note that there are no explicit constituent weights in the existing unstructured \ac{lta} approaches (i.e.\ all $w_{G, i} = 1$) because the constituents are explicitly given and assumed to be similarly indicative of $y_G$.
		For structured data however, where the decomposition $\varphi(G)$ is not given as part of the input, this assumption does not necessarily hold.
	\item \textbf{Aggregation:}
		Lastly a weighted aggregation function $\mathcal{A}: {(\mathcal{Y} \times \mathbb{R})}^{*} \to \mathcal{Y}$ must be learned.
		It combines the multiset of local constituent predictions and weights into a single global composition prediction.
\end{enumerate}

\section{An \acs*{lta} Interpretation of Existing \acsp*{gcnn}}%
\label{sec:ltag:interpretation}

\section{A Novel \acs*{lta}-Inspired \acs*{gcnn} Architecture}%
\label{sec:ltag:wl2gnn}
