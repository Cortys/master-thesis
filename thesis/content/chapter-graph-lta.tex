%!TEX root = ../main.tex
% chktex-file 46
\chapter{Learning to Aggregate on Graphs}%
\label{sec:ltag}

In the previous chapter an introduction to two separate fields of research was given:
\begin{enumerate*}
	\item \Acf{lta},
	\item \Acf{gcr}
\end{enumerate*}.
In this chapter we will combine them and define an extension of \ac{lta} to the \ac{gcr} problem.
This will be done in three steps:
\begin{enumerate}
	\item We begin with a formal definition of what actually constitutes an \ac{lta} method as opposed to non-\acs{lta} methods.
	\item Using this definition, we will see that some of the previously described \ac{gcnn} methods can be interpreted as \ac{lta} variants under certain conditions.
	\item Finally a new \acs{lta}-inspired \ac{gcnn} architecture will be described.
\end{enumerate}

\section{A Generalized Definition of \acs*{lta}}%
\label{sec:ltag:definition}

In order to formally define \ac{lta}, we must first decide on its defining characteristic.
We propose that this characteristic should be the \textit{localized explainability} of \ac{lta} predictions.
As seen in \cref{sec:related:lta}, an \ac{lta} score $y_{C} \in \mathcal{Y}$ for some multiset composition $C = \ldblbrace c_1, \dots, c_n \rdblbrace$ can always be tracked back to a set of local constituent scores $y_1, \dots, y_n \in \mathcal{Y}$.
Under the assumption that each constituent $c_i$ represents some human interpretable object, a composition's score $y_{C}$ can therefore be explained by the presence of certain indicative constituents/objects $c_i$.

Based on this intuition we now give a generalized definition of \ac{lta} which applies to unstructured as well as structured input data.
We assume that all compositions are represented by graphs $G \in \mathcal{G}$;
an unstructured input is represented by a graph with one vertex per constituent ($\mathcal{V}_G = \{ v_{c_1}, \dots, v_{c_n} \}$) and no edges ($\mathcal{E}_G = \emptyset$).
Each composition has some target score $y_G \in \mathcal{Y}$ which could be a discrete class or continuous value.
An \textit{\ac{lta} model} $h: \mathcal{G} \to \mathcal{Y}$ assigns predictions $\hat{y}_G \in \mathcal{Y}$ to compositions $G$ which ideally correspond to the true score $y_G$.
Such a model must satisfy three criteria:
\begin{enumerate}[label=\textbf{\arabic*.}]
	\item \textbf{Decomposition:}
		A given composition $G$ must be decomposed into a set of constituents $c_{G,i}$ via a \textit{decomposition function} $\varphi$.
		\begin{defn}
			$\varphi: \mathcal{G} \to \mathcal{P}(\mathcal{G})$ is a \textit{decomposition function} iff.\ it splits a graph into a subset of its subgraphs, i.e.\ $\forall G \in \mathcal{G}: \forall c_{G,i} \in \varphi(G): \exists s \in \mathcal{V}_G^{*}: c_{G,i} \equiv G[s]$.
		\end{defn}

		In the existing unstructured \ac{lta} approaches the decomposition function is implicitly defined as $\varphi(G) \coloneqq {\{ G[v_{c_i}] \}}_{v_{c_i} \in \mathcal{V}_G}$ since each vertex $v_{c_i}$ corresponds to an interpretable constituent $c_i$ by definition.
		For structured data however, a split into individual vertices is typically not appropriate.
		Molecular graphs from chemical datasets for example are meaningfully characterized by the presence of so-called \textit{functional groups} consisting of multiple bonded atoms while a characterization on the level of individual atoms is generally less meaningful~\cite{McNaught1997}.
	\item \textbf{Disaggregation:}
		The constituents $c_{G, i} \in \varphi(G)$ must be evaluated via some function $f: \mathcal{G} \to \mathcal{Y} \times \mathbb{R}$.
		This \textit{evaluation function} assigns a prediction $\hat{y}_{G, i} \in \mathcal{Y}$ and a weight $w_{G, i} \in \mathbb{R}$ to each constituent.
		A constituent's weight $w_{G, i}$ can intuitively be interpreted as a measure of the confidence that the local prediction $\hat{y}_{G, i}$ is indicative of the composition's global target score $y_G$.
		Learning local predictions and weights for all possible constituents is called the \textit{disaggregation problem}.

		Note that there are no explicit constituent weights in the existing unstructured \ac{lta} approaches (i.e.\ implicitly all $w_{G, i} = 1$) because the explicitly given constituents are assumed to be equally indicative of $y_G$.
		For structured data however, where the decomposition $\varphi(G)$ is not given as part of the input, this assumption does not necessarily hold.
		By weighting the constituents, an \ac{lta} model can reduce the relevance or even ignore constituents that turn out to be irrelevant for the compositions target score $y_G$.
	\item \textbf{Aggregation:}
		Lastly a \textit{weighted aggregation function} $\mathcal{A}: {(\mathcal{Y} \times \mathbb{R})}^{*} \to \mathcal{Y}$ must be applied.
		It combines the multiset of local constituent predictions and weights into a single global composition prediction.
		\begin{defn}\label{defn:ltag:weighted-agg}
			We call $\mathcal{A}$ a \textit{weighted aggregation function} iff.\ it satisfies
			\begin{align*}
				\text{idempotency: } & \forall y \in \mathcal{Y}, w \in \mathbb{R}_{> 0}^n: \mathcal{A}({\ldblbrace (y, w_i) \rdblbrace}_{i=1}^n) = y \\
				\land\text{ zero invariance: } & \forall y_0 \in \mathcal{Y}, S = {\ldblbrace (y_i, w_i) \rdblbrace}_{i=1}^n: \mathcal{A}(S \cup \{ (y_0, 0) \}) = \mathcal{A}(S) \text{.}
			\end{align*}
		\end{defn}
		One such weighted aggregator is the \textit{weighted mean} $\mean_w({\ldblbrace (y_i, w_i) \rdblbrace}_{i=1}^n) \coloneqq \sum_{i=1}^n w_i y_i$ which requires $\sum w_i = 1$ and $w_i \in [0, 1]$.
		Alternatively an unweighted aggregation function like $\min$, $\max$ or \ac{owa} also trivially satisfies \cref{defn:ltag:weighted-agg} if all inputs $y_i$ with $w_i = 0$ are filtered out and the weights for all remaining inputs are ignored.
\end{enumerate}
Based on the notion of decomposition, disaggregation and aggregation we can now define the concept of \textit{\ac{lta} formulations}.
\begin{defn}
	A model $h: \mathcal{G} \to \mathcal{Y}$ is in an \textit{\ac{lta} formulation} iff.\ it is expressed as
	\begin{align*}
		h(G) \coloneqq \mathcal{A}(\ldblbrace f(c_{G,i})\, |\, c_{G,i} \in \varphi(G) \rdblbrace) \quad\text{with $\varphi$, $f$ and $\mathcal{A}$ as defined above.}
	\end{align*}
\end{defn}
Note that every model $h: \mathcal{G} \to \mathcal{Y}$ has a trivial recursive \ac{lta} formulation by choosing $\varphi(G) = \{ G \}$, $f(G) = (h(G), 1)$ and an arbitrary weighted aggregation function $\mathcal{A}$.
Those trivial \ac{lta} formulations do not split compositions into locally evaluated constituents and therefore do not fulfill the localized explainability characteristic of \ac{lta}.

\section{An \acs*{lta} Interpretation of Existing \acsp*{gcnn}}%
\label{sec:ltag:interpretation}

\section{A Novel \acs*{lta}-Inspired \acs*{gcnn} Architecture}%
\label{sec:ltag:wl2gnn}
