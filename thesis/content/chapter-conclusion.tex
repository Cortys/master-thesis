%!TEX root = ../main.tex
% chktex-file 46
\chapter{Conclusion}%
\label{sec:conclusion}

To conclude the thesis, we now look back on the three research questions described in \cref{sec:intro:questions} and summarize the answers we gave to them in the previous chapters.
Afterwards, a brief overview of future research directions based on our findings will be given.

\section{Review}%
\label{sec:conclusion:review}

\paragraph{\circled{1}\; What constitutes an \ac{lta} method?}
We began with a general definition of \ac{lta} in \cref{sec:ltag:definition}.
There we proposed that its defining characteristic should be the \textit{localized explainability} of its predictions.
This characteristic was formalized via the notion of \textit{\ac{lta} formulations} (see \cfullref{defn:ltag:formulation}) which requires that a model is expressible in terms of a decomposition function $\psi: \mathcal{G} \to \mathcal{P}(\mathcal{G})$, a local evaluation function $f: \mathcal{G} \to \mathcal{Y} \times \mathcal{R}$ and a weighted aggregation function $\mathcal{A}: {(\mathcal{Y} \times \mathcal{R}_{\geq 0})}^* \to \mathcal{Y}$.
An ideal \ac{lta} method has such a formulation with a decomposition function $\psi$ that splits graphs into ``meaningful'' constituents in some domain-specific sense of the word.
Since this ideal notion of \ac{lta} is generally quite fuzzy, we only distinguished between \acs{lta}-like and non-\acs{lta} methods in this thesis;
a method was called non-\acs{lta} if it uses a trivial decomposition function that just splits a graph $G$ into the single ``constituent'' $G$.

\paragraph{\circled{2}\; How do existing \ac{gcr} methods relate to \ac{lta}?}
In \cref{sec:ltag:formulation:svm,sec:ltag:formulation:gcnn} we used our definition of \ac{lta} to check which of the existing \ac{gcr} approaches are compatible with it.
For the case of an \ac{svm} using a graph kernel/embedding we found that it is an \acs{lta}-like method if the kernel is a so-called nontrivial \acf{sce} (see \cfullref{defn:ltag:substruct-embedding,thm:ltag:svm-ltag-formulation}).
This \ac{sce} condition is satisfied by fingerprint embeddings, the \ac{wl} subtree kernel and the 2-LWL kernel which makes them \acs{lta}-like.
However, \texttt{graph2vec} embeddings, the \ac{wl} shortest-path kernel and the 2-GWL kernel were found to be trivial or only partly nontrivial \acp{sce}, i.e.\ they are non-\acs{lta} methods.
After considering those embedding approaches we looked at \ac{gcnn} and showed that they also have an \ac{lta} formulation under certain conditions (see \cfullref{thm:ltag:gcnn-ltag-formulation}).
More specifically, we saw that the constituents used by a \ac{gcnn} are spanned by the \ac{bfs} subtrees of its input graph.

\paragraph{\circled{3}\; What are limitations of existing graph \ac{lta} methods and how can they be overcome?}
\Cref{sec:ltd:edge-filter} described that the subtree constituents are their primary limitation of \acp{gcnn} and that more flexible decompositions can be learned via an edge filtering strategy.
To realize edge filtering, we proposed that informative edge feature vectors could be used as the input to a filtering classifier.
To produce such feature vectors we first looked at 2-\acsp{gnn} and found that they have various theoretical limitations, i.e.\ the inability to distinguish regular graphs and to detect cycles (see \cfullref{prop:ltd:2gnn-regular-limit,prop:ltd:2gnn-cycle-limit}).
We therefore proposed the 2-\acs{wl}-\acs{gnn} which does not have those limitations (see \cfullref{cor:ltd:wl2-gnn-regular}).

\paragraph{Evaluation results}
For the evaluation of our results we considered two aspects:
Firstly, we looked at how 2-\acs{wl}-\acsp{gnn} compare to other \acp{gnn}.
Secondly, we evaluated how \acs{lta}-like methods compare to non-\acs{lta} methods.
Regarding the first aspect, we showed that the theoretical advantages of 2-\acs{wl}-\acsp{gnn} are clearly observable on the synthetic triangle detection dataset while on the evaluated real-world datasets we got results which are generally comparable with the best state-of-the-art approaches but not significantly better.
Regarding the second aspect, we observed no general advantage or disadvantage of \acs{lta}-like methods.
While the \acs{lta}-like configurations of 2-\acs{wl}-\acsp{gnn} generally performed worse than their non-\ac{lta} counterparts, the \acs{lta}-like \ac{wl} subtree kernel generally performed quite well.
This shows that \ac{lta} is in principle suitable for graph classification tasks if the right decomposition, evaluation and aggregation functions are chosen.

\section{Future Directions}%
\label{sec:conclusion:todo}
