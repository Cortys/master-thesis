%!TEX root = ../main.tex
% chktex-file 46
\chapter{Conclusion}%
\label{sec:conclusion}

To conclude the thesis, we now look back on the three research questions described in \cref{sec:intro:questions} and summarize the answers that were given to them in the previous chapters.
Afterwards, a brief overview of future research directions based on our findings will be given.

\section{Review}%
\label{sec:conclusion:review}

\paragraph{\circled{1}\; What constitutes an \ac{lta} method?}
We began with a general definition of \ac{lta} in \cref{sec:ltag:definition}.
There we proposed that its defining characteristic is the \textit{localized explainability} of its predictions.
This characteristic was formalized via the notion of \textit{\ac{lta} formulations} (see \cfullref{defn:ltag:formulation}) which requires that a model is expressible in terms of a decomposition function $\psi: \mathcal{G} \to \mathcal{P}(\mathcal{G})$, a local evaluation function $f: \mathcal{G} \to \mathcal{Y} \times \mathbb{R}$ and a weighted aggregation function $\mathcal{A}: {(\mathcal{Y} \times \mathbb{R}_{\geq 0})}^* \to \mathcal{Y}$.
An ideal \ac{lta} method has such a formulation with a decomposition function $\psi$ that splits graphs into ``meaningful'' constituents in some domain-specific sense of the word.
Since this ideal notion of \ac{lta} is generally quite fuzzy, we only distinguished between \acs{lta}-like and non-\acs{lta} methods in this thesis;
a method was called non-\acs{lta} if it uses a trivial decomposition function that just splits a graph $G$ into the single global constituent $G$.

\paragraph{\circled{2}\; How do existing \ac{gcr} methods relate to \ac{lta}?}
In \cref{sec:ltag:formulation:svm,sec:ltag:formulation:gcnn} we used our definition of \ac{lta} to check which of the existing \ac{gcr} approaches are compatible with it.
For the case of an \ac{svm} using a graph kernel/embedding, we found that it is an \acs{lta}-like method if the kernel is a so-called nontrivial \acf{sce} (see \cfullref{defn:ltag:substruct-embedding,thm:ltag:svm-ltag-formulation}).
This \ac{sce} condition is satisfied by fingerprint embeddings, the \ac{wl} subtree kernel and the 2-LWL kernel, which makes them \acs{lta}-like.
However, \texttt{graph2vec} embeddings, the \ac{wl} shortest path kernel and the 2-GWL kernel were found to be trivial or only partly nontrivial \acp{sce}, i.e.\ they are non-\acs{lta} methods.
After considering those embedding approaches we looked at \acp{gcnn} and showed that they also have an \ac{lta} formulation under certain conditions (see \cfullref{thm:ltag:gcnn-ltag-formulation}).

\paragraph{\circled{3}\; What are limitations of existing graph \ac{lta} methods and how can they be overcome?}
While graph embeddings techniques can, in principle, decompose a given graph into arbitrary constituents (see fingerprint embeddings), the \ac{lta} formulation of \acp{gcnn} uses constituents that are spanned by the \ac{bfs} subtrees of its input graph.
In \cref{sec:ltd:edge-filter} we addressed this limitation of \acp{gcnn} by proposing that more flexible decompositions can be learned via an edge filtering strategy.
The idea behind this is to selectively prune subtree constituents by learning to ignore certain edges based on informative edge feature vectors.
To produce such edge feature vectors, we first looked at 2-\acsp{gnn} and found that they have various theoretical limitations, i.e.\ the inability to distinguish regular graphs and to detect cycles (see \cfullref{prop:ltd:2gnn-regular-limit,prop:ltd:2gnn-cycle-limit}).
We therefore proposed the novel 2-\acs{wl}-\acs{gnn} which does not have those limitations (see \cfullref{cor:ltd:wl2-gnn-regular}).
It additionally also has a strictly greater expressive power than all the 1-\acs{wl} bounded vertex neighborhood aggregation \acp{gnn}, e.g.\ \ac{gcn} or \ac{gin} (see \cfullref{cor:ltd:wl2-gnn-more-wl1-power}).

\paragraph{Evaluation results}
For the evaluation of our results we considered two aspects:
Firstly, we looked at how 2-\acs{wl}-\acsp{gnn} compare to other \acp{gnn}.
Secondly, we evaluated how \acs{lta}-like methods compare to non-\acs{lta} methods.
Regarding the first aspect, we showed that the theoretical advantages of 2-\acs{wl}-\acsp{gnn} are clearly observable on the synthetic triangle detection dataset while on the evaluated real-world datasets we got results which are generally comparable with the best state-of-the-art approaches but not significantly better.
Regarding the second aspect, we observed no general advantage or disadvantage of \acs{lta}-like methods.
While the \acs{lta}-like configurations of 2-\acs{wl}-\acsp{gnn} generally performed worse than their non-\ac{lta} counterparts, the \acs{lta}-like \ac{wl} subtree kernel generally performed quite well.
This shows that \ac{lta} is in principle suitable for graph classification tasks if the right decomposition, evaluation and aggregation functions are chosen.

\section{Future Directions}%
\label{sec:conclusion:todo}

Based on our results, let us now consider which follow-up questions could be tackled in future research.

\paragraph{Further theoretical analysis of 2-\acs{wl}-\acsp{gnn}}
Even though we proved that our proposed 2-\acs{wl} inspired convolution approach has certain theoretical advantages over the existing methods, we did not fully characterize its expressive and computational power w.r.t.\ the 2-\acs{wl} algorithm itself.
As we empirically showed in \cref{sec:eval:synthetic}, 2-\acs{wl}-\acsp{gnn} are able to learn to detect certain triangles in graphs.
We did however not consider the aspect of $m$-cycle detection or counting for $m > 3$.
Further work is required to adapt the theoretical results for the 2-\acs{wl} algorithm to the graph convolution setting.
One particularly interesting question there would be to analyze how exactly the neighborhood radius $r$ relates to the expressive power of a 2-\acs{wl}-\acs{gnn}.

\paragraph{Further empirical real-world evaluation of 2-\acs{wl}-\acsp{gnn}}
In our evaluations on real-world datasets we did not observe a clear advantage of 2-\acs{wl}-\acsp{gnn} over other methods.
We did however find indicators for the hypothesis that 2-\acs{wl}-\acsp{gnn} are able to perform better on datasets containing graphs with cycles when increasing the neighborhood radius.
The cycle detection abilities of 2-\acs{wl} could therefore potentially be relevant for real-world problems.
Future research could evaluate 2-\acs{wl}-\acsp{gnn} on a more diverse set of problem domains, e.g.\ software analysis on \acfp{cfg}, and with other pooling layers to determine to which extent the theoretical advantages of 2-\acs{wl}-\acsp{gnn} can lead to an advantage in practice.

\paragraph{Evaluation of the explainability of \acs{lta}-like models}
Our definition of \ac{lta} was motivated by the idea that the local evaluations of constituents provide an explanation for the global aggregated prediction.
However, in our evaluation of \ac{lta} we focused mostly on the question how \acs{lta}-like approaches perform compared to non-\acs{lta} approaches, with the result that \acs{lta}-like models does not appear to have a general advantage or disadvantage regarding classification accuracy.
In the next step it should be evaluated whether the constituent scores of an \acs{lta}-like model are meaningful and therefore provide an advantage from the perspective of \acl{xai}.

\paragraph{Solving the \ac{ltd} problem via edge filtering}
The motivation behind the proposed 2-\acs{wl}-\acs{gnn} was to use it to obtain informative edge feature vectors with which an edge filter could be trained to dynamically decompose graphs.
The results in this thesis provide the first step towards the realization of this idea.
Further research is required to determine whether this approach is suitable to learn meaningful constituents in real-world problem domains.
