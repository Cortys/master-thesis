%!TEX root = ../main.tex
% chktex-file 46
\chapter{Evaluation}%
\label{sec:eval}

We will now empirically evaluate the ideas presented in the previous chapters.
In \cref{sec:ltag} the relation between \ac{lta} and existing \ac{gcr} approaches was formally analyzed.
There we saw that the \acs{lta}-formulations of existing approaches mostly use static decomposition functions, e.g.\ \acs{bfs}-subtree decompositions.
Motivated by the idea of dynamically learning decompositions via edge filters, we proposed the novel 2-\acs{wl}-\acs{gnn} in the last chapter.
Note however that a 2-\acs{wl}-\acs{gnn} by itself is not more ``\acs{lta}-like'' than other \acp{gnn}.

In order to empirically evaluate the theoretical results of this thesis, we therefore differentiate between two independent evaluation goals:
\begin{enumerate}[label={\textbf{\arabic*.}}]
	\item \textbf{Comparison of 2-\acs{wl}-\acsp{gnn} with other \ac{gnn} methods:}
		Even though it was motivated by \ac{lta}, the 2-\acs{wl} convolution layer should be compared with other convolution approaches in a non-\acs{lta} configuration.
	\item \textbf{Comparison of \acs{lta}-like \ac{gcr} methods with non-\acs{lta} methods:}
\end{enumerate}

\section{Experimental Setup}%
\label{sec:eval:setup}

\section{Synthetic Datasets}%
\label{sec:eval:synthetic}

\section{Real-World Datasets}%
\label{sec:eval:real}
