%!TEX root = ../main.tex
% chktex-file 46

\chapter{Appendix}%
\label{sec:appendix}

\section{Evaluated Hyperparameter Grids}%
\label{sec:appendix:config-grid}

\section{Dataset Statistics and Descriptions}%
\label{sec:appendix:ds-stats}

\begin{table}[ht]
	\caption{Sizes of the evaluated binary classification datasets and their graphs.}\label{tbl:appendix:ds-stats}
	\centering
	\csvreader[
		tabular={lrrrrrrrr},
		separator=comma,
		table head={%
			\multicolumn{1}{c}{} & \multicolumn{1}{c}{} & &\multicolumn{3}{c}{vertex count $\left|\mathcal{V}_G\right|$} & \multicolumn{3}{c}{edge count $\left|\mathcal{E}_G\right|$} \\%
			& \multirow{-2}{*}[-0.2em]{\shortstack[c]{no.\ of\\ graphs}} & \multirow{-2}{*}[-0.2em]{\shortstack[c]{vertex data\\(feat.\ + lab.)}} & $\min$ & $\mean$ & $\max$ & $\min$ & $\mean$ & $\max$ \\\toprule%
		},
		table foot=\bottomrule,
		late after line=\\
	]{data/ds_stats.csv}%
	{name=\name,graph_count=\gcount,%
	node_count_min=\ncountmin,node_count_mean=\ncountmean,node_count_max=\ncountmax,%
	edge_count_min=\ecountmin,edge_count_mean=\ecountmean,edge_count_max=\ecountmax,%
	node_degree_min=\ndegmin,node_degree_mean=\ndegmean,node_degree_max=\ndegmax,%
	dim_node_features=\nfdim,dim_edge_features=\efdim%
	}%
	{\textbf{\name}&%
	$\gcount$&%
	$\nfdim$&%
	$\ncountmin$&$\ncountmean$&$\ncountmax$&%
	$\ecountmin$&$\ecountmean$&$\ecountmax$%
	}
\end{table}

\paragraph{TRIANGLE}
The triangle detection dataset was generated by sampling three graphs with exactly one unicolored triangle uniformly at random for each possible combination of the following parameters:
The number of vertices (between 6 and 32), the vertex color proportions (either 50/50\%, 75/25\% or 25/75\% vertices with the colors \colorlabel{t_blue}{A}/\colorlabel{t_red}{B}), the graph density (${\left| \mathcal{V}_{G} \right|}^{-2} \left| \mathcal{E}_{G} \right| \in \{ \nicefrac{1}{4}, \nicefrac{1}{2} \}$) the graph class (add a triangle with either the color \colorlabel{t_blue}{A} or \colorlabel{t_red}{B}).

\paragraph{NCI1}
This dataset was made available by \citet{Shervashidze2011}.
It contains a balanced subset of molecule graphs that were originally published by the US \ac{nci}~\cite{Wale2007}.
In each molecule graph, vertices correspond to atoms and edges to bonds between them.
The binary classes in this dataset describe whether a molecule is able to suppress or inhibit the growth of certain lung cancer and ovarian cancer cell lines in humans.

\paragraph{PROTEINS and D\&D}
The graphs in the PROTEINS dataset represent proteins~\cite{Borgwardt2005a}.
Each vertex corresponds to a so-called \ac{sse}, i.e.\ a certain molecular substructure.
An edge encodes either that two \acp{sse} are neighbors in the protein's amino-acid sequence or that those \acp{sse} are close to each other in 3D space.
Each protein graph is classified by whether it is an enzyme or not.
Very similarly, the D\&D dataset~\cite{Dobson2003} is also about classifying proteins as enzymes or non-enzymes but it uses a different selection of vertex features.

\paragraph{REDDIT}
This balanced dataset contains graphs that represent online discussion threads on the website Reddit~\cite{Yanardag2015}.
Each vertex corresponds to a user; an edge is drawn between two users iff.\ at least one of them replied to a comment of another.
Such social interaction graphs were sampled from two types of subreddits:
Question/answer-based and discussion-based.
The classification goal is to predict from which type of subreddit a given graph was sampled.

\paragraph{IMDB}
This dataset contains so-called \textit{ego-networks} of movie actors~\cite{Yanardag2015}.
Vertices in such networks represent actors and edges encode whether two actors starred in the same movie.
The graphs in the dataset are derived from the actors starring in either action or romance movies.
The classification goal for each graph is to predict the movie genre it was derived from.

\section{Fold-wise Accuracy Deltas}%
\label{sec:appendix:fold-diffs}
