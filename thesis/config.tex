% !TEX root = main.tex
% chktex-file 46

% **************************************************
% Files' Character Encoding
% **************************************************
\PassOptionsToPackage{utf8}{inputenc}
\usepackage{inputenc}
\usepackage[ngerman,english]{babel}

% **************************************************
% Information and Commands for Reuse
% **************************************************
\newcommand{\thesisTitle}{Learning to Aggregate on Structured Data}
\newcommand{\thesisName}{Clemens Damke}
\newcommand{\thesisMatNr}{7011488}
\newcommand{\thesisSubject}{Master Thesis}
\newcommand{\thesisDate}{\today}
\newcommand{\thesisVersion}{Draft}

\newcommand{\thesisFirstReviewer}{Prof.~Dr.~Eyke Hüllermeier}
\newcommand{\thesisFirstReviewerUniversity}{Paderborn University}
\newcommand{\thesisFirstReviewerDepartment}{Intelligent Systems and Machine Learning Group (ISG)}

\newcommand{\thesisSecondReviewer}{Prof.~Dr.~Axel-Cyrille Ngonga Ngomo}
\newcommand{\thesisSecondReviewerUniversity}{Paderborn University}
\newcommand{\thesisSecondReviewerDepartment}{Data Science Group (DICE)}

\newcommand{\thesisSupervisor}{Vitalik Melnikov}

\newcommand{\thesisUniversity}{Paderborn University}
\newcommand{\thesisUniversityDepartment}{Department of Electrical Engineering, Computer Science and Mathematics}
\newcommand{\thesisUniversityInstitute}{Heinz Nixdorf Institute}
\newcommand{\thesisUniversityGroup}{Intelligent Systems and Machine Learning Group (ISG)}
\newcommand{\thesisUniversityStreetAddress}{Warburger Straße 100}
\newcommand{\thesisUniversityPostalCode}{33098}
\newcommand{\thesisUniversityCity}{Paderborn}


% **************************************************
% Debug LaTeX Information
% **************************************************
%\listfiles


% **************************************************
% Load and Configure Packages
% **************************************************

% Colors:
\usepackage[usenames, dvipsnames, svgnames, table]{xcolor}

\definecolor{t_blue}{HTML}{355fb3}
\definecolor{t_red}{HTML}{b33535}
\definecolor{t_green}{HTML}{3bb335}
\definecolor{t_yellow}{HTML}{b39735}
\definecolor{t_darkblue}{HTML}{1e3666}
\definecolor{t_darkgreen}{HTML}{22661e}
\definecolor{t_darkyellow}{HTML}{66571e}
\definecolor{t_lightblue}{HTML}{8ea7d7}

% Code snippets:
\usepackage{minted}
\usepackage{etoolbox,xpatch}
\makeatletter
\AtBeginEnvironment{minted}{\dontdofcolorbox}
\def\dontdofcolorbox{\renewcommand\fcolorbox[4][]{##4}}
\xpatchcmd{\inputminted}{\minted@fvset}{\minted@fvset\dontdofcolorbox}{}{}
\makeatother
\setminted{
	fontsize=\footnotesize,
	numbers=left,
	tabsize=4,
	breaklines=true
}

\PassOptionsToPackage{% setup clean thesis style
    figuresep=space,
    sansserif=false,
    hangfigurecaption=false,
    hangsection=true,
    hangsubsection=true,
    colorize=full,
    colortheme=custom,
	colormain=t_darkblue,
	coloraccessory=t_blue,
    bibsys=bibtex,
    bibfile=literature,
    bibstyle=alphabetic,
    wrapfooter=false,
}{cleanthesis}
\usepackage{cleanthesis}

\usepackage{mathtools}
\usepackage{amssymb}
\usepackage{amsthm}
\usepackage{thmtools}
\usepackage{bm}
\usepackage{bbm}
\usepackage{dsfont}
\usepackage{centernot}
\usepackage{breqn}
\usepackage{nicefrac}
\newcommand\numberthis{\addtocounter{equation}{1}\tag{\theequation}}
\newcommand*{\dblbrace}[2][]{#1{#2}\ifthenelse{\equal{#1}{}}{\mskip-6mu}{\mskip-8mu}#1{#2}}
\newcommand*{\ldblbrace}[1][]{\dblbrace[#1]{\{}}
\newcommand*{\rdblbrace}[1][]{\dblbrace[#1]{\}}}
\newcommand*\dif{\mathop{}\!\mathrm{d}}

\makeatletter
\renewenvironment{proof}[1][\proofname]{\par
	\pushQED{\qed}%
	\topsep-10pt
	\trivlist%
	\item[\hskip\labelsep% chktex 41
		\itshape%
		#1\@addpunct{.}%
	]\ignorespaces%
}{%
  \popQED\endtrivlist\@endpefalse% chktex 21
}
\def\th@plain{\thm@preskip\parskip\thm@postskip0pt\itshape} % chktex 6
\def\th@definition{\thm@preskip\parskip\thm@postskip0pt\normalfont}
\def\th@remark{\thm@headfont{\itshape}\normalfont\thm@preskip\parskip\thm@postskip0pt} % chktex 6
\makeatother

\usepackage{pifont}
\usepackage{graphicx}
\usepackage{tikz}
\usetikzlibrary{arrows,positioning}
\usetikzlibrary{calc}

\usepackage{pgfplots}
\usepackage{pgfplotstable}
\pgfplotsset{compat=1.14}
\usepgfplotslibrary{dateplot, statistics}
\pgfplotsset{
    cycle list={t_blue\\t_red\\t_green\\},
}

\usepackage{listings}
\lstset{basicstyle=\ttfamily,breaklines=true}

\usepackage{tasks}
\settasks{counter-format=tsk[1].}

\usepackage{acro}
\acsetup{first-long-format=\slshape} % chktex 6
\acsetup{single}
\acsetup{use-barriers}
\acsetup{reset-at-barriers}

\usepackage{stmaryrd}
\usepackage{multicol}
\usepackage{pbox}
\usepackage{longtable}
\usepackage{booktabs}
\usepackage{csvsimple}
\usepackage{siunitx}
\usepackage[nameinlink]{cleveref}

\hypersetup{% setup the hyperref-package options
    pdftitle={\thesisTitle},    %   - title (PDF meta)
    pdfsubject={\thesisSubject},%   - subject (PDF meta)
    pdfauthor={\thesisName},    %   - author (PDF meta)
    plainpages=false,           %   -
    colorlinks=false,           %   - colorize links?
    pdfborder={0 0 0},          %   -
    breaklinks=true,            %   - allow line break inside links
    bookmarksnumbered=true,     %
    bookmarksopen=true,         %
	hypertexnames=false,        %   - fix wrong page links across chapters
}

% Custom commands:
\newcommand{\tikzmark}[1]{\tikz[overlay,remember picture] \node (#1) {};} % chktex 1
\newcommand{\dac}[3]{\DeclareAcronym{#1}{short = #2, long = #3}}
\newcommand*\circled[2][1pt]{\tikz[baseline=(char.base)]{ % chktex 36
    \node[shape=circle,draw,inner sep=#1] (char) {#2};}}
\newcommand*{\badgeboxinline}[2][black]{\fcolorbox{#1}{white}{\textsf{\small\textcolor{#1}{\,#2\,}}}}
\newcommand*{\badgebox}[2][black]{\null\hfill\badgeboxinline[#1]{#2}}
\newcommand{\cmark}{\ding{51}}
\newcommand{\xmark}{\ding{55}}

\newcommand{\sourceinline}[2][source]{{\scriptsize\textsc{#1:~\cite{#2}}}}
\newcommand{\source}[2][source]{\null\hfill\sourceinline[#1]{#2}}
\newcommand{\cfullref}[2][, ]{\cref{#2}#1\cpageref{#2}}

\DeclareMathOperator{\Tr}{Tr}
\DeclareMathOperator{\mean}{mean}
\DeclareMathOperator{\wmean}{wmean}
\DeclareMathOperator{\wmaj}{wmaj}
\DeclareMathOperator{\sgn}{sgn}
\DeclareMathOperator{\set}{set}

\theoremstyle{plain}
\newtheorem{thm}{Theorem}
\numberwithin{thm}{chapter}
\newtheorem{prop}[thm]{Proposition}
\numberwithin{prop}{chapter}
\newtheorem{lem}[thm]{Lemma}
\numberwithin{lem}{chapter}
\newtheorem{fact}[thm]{Fact}
\numberwithin{fact}{chapter}
\newtheorem{cor}[thm]{Corollary}
\numberwithin{cor}{chapter}

\theoremstyle{definition}
\newtheorem{defn}[thm]{Definition}
\numberwithin{defn}{chapter}
