% chktex-file 46
\documentclass[12pt]{scrartcl}

\PassOptionsToPackage{utf8}{inputenc}
\usepackage{inputenc}
\usepackage[american]{babel}
\usepackage{csquotes}
\usepackage{microtype}

\usepackage{graphicx}
\graphicspath{{images/}}

\usepackage{paralist}
\usepackage{csquotes}
\usepackage[T1]{fontenc}
\usepackage{lmodern}

\usepackage{geometry}
% \geometry{a4paper,body={5.8in,9in}}
\geometry{a4paper}
\renewcommand{\baselinestretch}{1.1}
\usepackage{parskip}
\setlength{\parindent}{0pt}

\usepackage{amsmath, amsfonts, amssymb}
\usepackage{bm}
\usepackage{placeins}
\usepackage{subcaption}

\usepackage{setspace}

\usepackage{hyperref}
\usepackage[nameinlink]{cleveref}
\newcommand{\crefrangeconjunction}{--}

\usepackage[						% use biblatex for bibliography
	backend=bibtex,					% 	- use biber backend (bibtex replacement) or bibtex
	style=numeric,					% 	- use alphabetic (or numeric) bib style
	natbib=true,					% 	- allow natbib commands
	hyperref=true,					% 	- activate hyperref support
	backref=true,					% 	- activate backrefs
	isbn=false,						% 	- don't show isbn tags
	url=false,						% 	- don't show url tags
	doi=false,						% 	- don't show doi tags
	urldate=long,					% 	- display type for dates
	maxnames=3,%
	minnames=1,%
	maxbibnames=5,%
	minbibnames=3,%
	maxcitenames=2,%
	mincitenames=1%
]{biblatex}
\bibliography{literature}

\usepackage[inline]{enumitem}

%%%%%%%%%%%%%%%%%%%%%%%%%%%%%%%%%%%%%%

\newcommand{\thesisTitle}{Learning to Aggregate on Structured Data}
\newcommand{\thesisSubject}{Master Thesis Proposal \& Work Plan}
\newcommand{\thesisName}{Clemens Damke}
\newcommand{\thesisMail}{cdamke@mail.uni-paderborn.de}
\newcommand{\thesisMatNr}{7011488}
\hypersetup{% setup the hyperref-package options
    pdftitle={\thesisTitle},    %   - title (PDF meta)
    pdfsubject={\thesisSubject},%   - subject (PDF meta)
    pdfauthor={\thesisName},    %   - author (PDF meta)
    plainpages=false,           %   -
    colorlinks=false,           %   - colorize links?
    pdfborder={0 0 0},          %   -
    breaklinks=true,            %   - allow line break inside links
    bookmarksnumbered=true,     %
    bookmarksopen=true          %
}

\begin{document}

\title{\thesisTitle}
\subtitle{\thesisSubject}
\author{{\thesisName}\\\small{Matriculation Number: \thesisMatNr}\\\small{\href{mailto:\thesisMail}{\thesisMail}}}
\date{\today}
\maketitle

\section{Motivation}%
\label{sec:motivation}

Most of the commonly used supervised machine learning techniques assume that instances are represented by $d$-dimensional feature vectors $x \in \mathcal{X} = \mathcal{X}_1 \times \cdots \times \mathcal{X}_d$ for which some target value $y \in \mathcal{Y}$ should be predicted.
In the regression setting the target domain $\mathcal{Y}$ is continuous, typically $\mathcal{Y} = \mathbb{R}$, whereas $\mathcal{Y}$ is some discrete set of classes in the classification setting.

Since not all data is well-suited for a fixed-dimensional vector representation, approaches that directly consider the structure of the input data might be more appropriate in such cases.
One such case is the class of so-called \textit{learning to aggregate} (LTA) problems as described by \citet{Melnikov2016}.
There the instances are represented by compositions $\bm{c}$ of constituents $c_i \in \bm{c}$, i.e.\@ variable-size multisets.
The assumption in LTA problems is that for all constituents $c_i$ a local valuation $y_i \in \mathcal{Y}$ is either given or computable.
The set of those local valuations should be indicative of the overall valuation $y \in \mathcal{Y}$ of the entire composition $\bm{c}$.
The goal of LTA is to learn a variadic aggregation function $A: \mathcal{Y}^{*} \to \mathcal{Y}$ that estimates such composite valuations, i.e.\@ $\hat{y} = A(y_1, \dots, y_n)$ for a composition with $n$ constituents.
Additionally the aggregation function $A$ should be associative and commutative to fit with the multiset-structure of compositions.

Current LTA approaches only work with multiset inputs.
In practice there is however often some relational structure among the constituents of a composition.
This effectively turns LTA into a graph classification or regression problem.
The overall aim of this thesis is to look into the question of how aggregation function learning methods might be generalized to the graph setting.

\section{Related Work}%
\label{sec:related-work}

This thesis will be based on two currently mostly unrelated fields of research:
\begin{enumerate*}
	\item Learning to Aggregate
	\item Graph classification
\end{enumerate*}.
A short overview of the current state-of-the-art approaches in both fields will be given now.

\subsection{Learning to Aggregate}%
\label{sec:related-work:lta}

Two main approaches to represent the aggregation function in LTA problems have been explored.
The first approach uses so-called uninorms~\cite{Melnikov2016} to do so.
There the basic idea is to express composite valuations as fuzzy logical assignments $y \in [0, 1]$.
Such a composite assignment $y$ is modeled as the result of a parameterized logical expression of constituent assignments $y_i \in [0, 1]$.
As the logical expression that thus effectively aggregates the constituents, a uninorm $U_{\lambda}$ is used.
Depending on the parameter $\lambda$, a uninorm interpolates between t-norms and t-conorms which correspond to logical conjunction and disjunction respectively.
\begin{align*}

\end{align*}

\subsection{Graph Classification}%
\label{sec:related-work:gc}

\section{Goals}%
\label{sec:goals}

\subsection{Required Goals}%
\label{sec:goals:req}

\subsection{Optional Goals}%
\label{sec:goals:opt}

\section{Approach}%
\label{sec:approach}

\section{Preliminary Document Structure}%
\label{sec:doc-structure}

\begin{enumerate}
	\item Introduction
	\item \dots
\end{enumerate}

\section{Time-Schedule}%
\label{sec:schedule}

\begin{figure}[!ht]
	\centering
	%\includegraphics[width=.9\textwidth]{timeschedule}
	\caption{Sketch of the time schedule for the work on the thesis}\label{fig:schedule}
\end{figure}

% % %
\newpage
{%
\renewcommand{\bibfont}{\normalfont\small}
\setlength{\biblabelsep}{5pt}
\setlength{\bibitemsep}{0.5\baselineskip plus 0.5\baselineskip} % chktex 1
\setcounter{biburllcpenalty}{9000}
\setcounter{biburlucpenalty}{9999}
\printbibliography%
}

\vspace{6cm}

\begin{center}
	\begin{tabular}{l p{0.1\textwidth} r}
		\cline{1-1} \cline{3-3}
		\begin{minipage}[t]{0.4\textwidth}
			\centering
			\vspace{0cm}Supervisor
		\end{minipage}
		&
		\begin{minipage}[t]{0.2\textwidth}
		\end{minipage}
		&
		\begin{minipage}[t]{0.4\textwidth}
			\centering
			\vspace{0cm}Student
		\end{minipage}
	\end{tabular}
\end{center}

\end{document}
